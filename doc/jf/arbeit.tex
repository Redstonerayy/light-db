% set font and paper size
\documentclass[11pt,a4paper]{article}

% change to german
\usepackage[german]{babel}

% better hyphenation
\usepackage[final]{microtype}
\usepackage{csquotes}

% packages, images, math
\usepackage{geometry, graphicx, amsmath, amsfonts, array, multicol, multirow}

% bar charts
\usepackage{bchart}

% for bibliography
\usepackage[
    backend=biber,
    style=alphabetic,
    sorting=ynt,
    minalphanames=3,
]{biblatex}
\addbibresource{./res/references.bib}

% remove dots in ToC
\usepackage{tocloft}

% import line spacing
\usepackage{setspace}

% colors
\usepackage{xcolor}

% spacing between section titles and text
\usepackage{titlesec}
\titlespacing*{\section}{0pt}{0.7ex}{0.7ex}
\titlespacing*{\subsection}{0pt}{0.7ex}{0.7ex}
\titlespacing*{\subsubsection}{0pt}{0.7ex}{0.7ex}

% for urls
% break urls on line end
\usepackage[breaklinks, colorlinks=true, urlcolor=blue, citecolor=blue, linkcolor=black]{hyperref}

% Remove Indentation at new line
\setlength{\parindent}{0cm}

% Set Font to Arial, needs xelatex
\usepackage{unicode-math}
\usepackage{fontspec}
\setmainfont{Arial}

% Set Layout
\geometry{
    a4paper,
    left=25mm,
    right=25mm,
    top=25mm,
    bottom=20mm
}

% plots
\usepackage{pgfplotstable}
% recommended:
\usepackage{booktabs}
\usepackage{array}
\usepackage{colortbl}

\usepackage{pgfplots}

% moving in tikz picture
\usepackage{changepage}

% position figures
\usepackage{float}

% set line spacing
\setstretch{1.5}

% remove ToC page number
\AtBeginDocument{\addtocontents{toc}{\protect\thispagestyle{empty}}} 

% bold caption
\usepackage[labelfont=bf]{caption}

% document
\begin{document}

% deckblatt
% title
\title{Schwierigkeiten bei Implementierung und Evaluation von Datenstrukturen in Datenbanksystemen}
\author{Anton Rodenwald (19)}
\maketitle

% page settings
\addtocounter{page}{-3}
\thispagestyle{empty}

\large
\begin{tabular}{l p{12cm}}

    Fachgebiet:          & Mathematik/Informatik                                            \\

    Wettbewerbssparte:   & Jugend Forscht                                                   \\

    Bundesland:          & Niedersachsen                                                    \\

    Wettbewerbsjahr:     & 2024                                                             \\

    Thema des Projektes: &
    In meinem Projekt wollte ich verschiedene in Datenbanksystemen genutzte Datenstrukturen
    implementieren und testen. Dabei betrachtete ich auch unterschiedliche Ansätze der
    Datenspeicherung und für den Datenzugriff. Da einige Schwierigkeiten auftraten,
    werde ich auch mögliche Stolpersteine und Lösungen erläutern. \\
 
    Projektbetreuerin:   & Birgit Ziegenmeyer                                               \\

    Institution:         & Schillerschule Hannover                                          \\
\end{tabular}

\clearpage

\pagestyle{empty}

% kurzfassung

\section*{Kurzfassung}

<Text>

\clearpage

% inhaltsverzeichnis

\renewcommand*\contentsname{Inhaltsverzeichnis}

\renewcommand{\cftdot}{}

\tableofcontents

\clearpage

\pagestyle{plain}

\section{Motivation, Fragestellung und Ziel}

Im Rahmen des Informatik Leistungkurses meiner Schule haben wir (Mitschüler und ich) uns
mit Datenbanken und Modellierung beschäftigt. Da jedoch auf die Funktionsweise einer
Datenbank nicht weiter eingegangen wurde und ich auch schon von Unterschiedlichen
Datenbankansätzen, genauer Relationalen und Nicht-Relationalen, gehört hatte,
wollte ich mich ihm Rahmen meines Projektes genauer damit beschäftigen.
Dazu wollte ich eine einfache Datenbank umsetzen.\newline

Meine Forschungsfrage ist deshalb, wie die Wahl der Datenstruktur die
Geschwindigkeit einer Datenbank beeinflusst und inwiefern sich eine Datenstruktur
für gewisse Daten besser oder schlechter eignet.\newline

Mein Ursprüngliches Ziel war dabei die Programmierung einer Datenbank
mit den Datenstrukturen R-Tree, B-Tree, Binary-Tree und Graphen.
Davon konnte ich leider aufgrund der hohen Komplexität vieles nicht umsetzen.

\section{Hintergrund}

\subsection{Arten von Daten}

\subsection{Betrachtete Datenstrukturen}

r-tree (spatial data)
b-tree
indices (normal tables)
graphs (search algos)

\subsection{Arten von Datenbanksystemen}

\section{Vorgehensweise}

% implementaton -> networking
\subsection{implementation -> welche algorithmen}

\section{schwierigkeiten/limitationen}

kein echtes server testing, weil hardware nicht so geeignet
AMDuProf hat nicht funktioniert, weil keine ahnung
% https://community.amd.com/t5/server-gurus-discussions/amduprof-not-attaching-to-processes-on-linux/m-p/578286#M1589

\section{messwerte}
\section{Ergebnisse}

\section{empfehlung von datenbanken -> einfach regeln}

\subsection{Beantwortung der Forschungsfrage}

\section{Gelernt}
gdb
b tree
socket api linux

\section{fazit}

\section{ausblick interessen geweckt}

% Quellen
% bjnet
% 

\clearpage

\begin{bchart}[min=0, max=30, scale=1.9]
    \bcbar[label=1, text=List Comprehension Python mit Konvertierung zu numba.typed list]{26.849}
    \smallskip
    \bcbar[label=2, text=\hspace{7cm}List Comprehension Python]{9.307}
    \smallskip
    \bcbar[label=3, text=\hspace{2cm}List Comprehension Python mit Numba]{1.148}
    \smallskip
    \bcbar[label=4, text=\hspace{2cm}NumPy np.random.randint mit Numba]{0.401}
    \smallskip
    \bcbar[label=5, text=\hspace{2cm}NumPy np.random.randint]{0.051}
    \smallskip
    \bcxlabel{\bf{Ausführungszeit in Sekunden}}
\end{bchart}

\clearpage

\begin{figure}[H]
    \centering
    \begin{tikzpicture}[scale=1]
        \begin{axis}[
                x label style={at={(axis description cs:0.5,-0.01)},anchor=north},
                y label style={at={(axis description cs:-0.01,0.5)},anchor=south},
                xlabel=\textbf{Nummer der Messung},
                ylabel=\textbf{Zeit in Nanosekunden},
                width=1.0\textwidth,
                height=0.7\textwidth,
            ]
            \addplot table [y=$SA_F$, x=NR]{./res/data_pc_1.dat};
            \addlegendentry{$SA_F$ series}
            \addplot table [y=$SA_H$, x=NR]{./res/data_pc_1.dat};
            \addlegendentry{$SA_H$ series}
            \addplot table [y=$G_F$, x=NR]{./res/data_pc_1.dat};
            \addlegendentry{$G_F$ series}
            \addplot table [y=$G_H$, x=NR]{./res/data_pc_1.dat};
            \addlegendentry{$G_H$ series}
        \end{axis}
    \end{tikzpicture}
    \vspace*{-0.8cm}
    \caption{\textbf{1. Messreihe am PC (Ryzen 7 2700)}}
\end{figure}

\begin{figure}[H]
    \centering
    \begin{tikzpicture}[scale=1]
        \begin{axis}[
                x label style={at={(axis description cs:0.5,-0.01)},anchor=north},
                y label style={at={(axis description cs:-0.01,0.5)},anchor=south},
                xlabel=\textbf{Nummer der Messung},
                ylabel=\textbf{Zeit in Nanosekunden},
                width=1.0\textwidth,
                height=0.7\textwidth,
            ]
            \addplot table [y=$SA_F$, x=NR]{./res/data_pc_2.dat};
            \addlegendentry{$SA_F$ series}
            \addplot table [y=$SA_H$, x=NR]{./res/data_pc_2.dat};
            \addlegendentry{$SA_H$ series}
            \addplot table [y=$G_F$, x=NR]{./res/data_pc_2.dat};
            \addlegendentry{$G_F$ series}
            \addplot table [y=$G_H$, x=NR]{./res/data_pc_2.dat};
            \addlegendentry{$G_H$ series}
        \end{axis}
    \end{tikzpicture}
    \vspace*{-0.8cm}
    \caption{\textbf{2. Messreihe am PC (Ryzen 7 2700)}}
\end{figure}

\clearpage

\begin{figure}[H]
    \centering
    \begin{tikzpicture}[scale=1]
        \begin{axis}[
                x label style={at={(axis description cs:0.5,-0.01)},anchor=north},
                y label style={at={(axis description cs:-0.01,0.5)},anchor=south},
                xlabel=\textbf{Nummer der Messung},
                ylabel=\textbf{Zeit in Nanosekunden},
                scaled ticks=false,
                width=1.0\textwidth,
                height=0.7\textwidth,
            ]
            \addplot table [y=$SA_F$, x=NR]{./res/data_laptop_1.dat};
            \addlegendentry{$SA_F$ series}
            \addplot table [y=$SA_H$, x=NR]{./res/data_laptop_1.dat};
            \addlegendentry{$SA_H$ series}
            \addplot table [y=$G_F$, x=NR]{./res/data_laptop_1.dat};
            \addlegendentry{$G_F$ series}
            \addplot table [y=$G_H$, x=NR]{./res/data_laptop_1.dat};
            \addlegendentry{$G_H$ series}
        \end{axis}
    \end{tikzpicture}
    \vspace*{-0.8cm}
    \caption{\textbf{1. Messreihe am Laptop (Ryzen 5 5500U)}}
\end{figure}

\begin{figure}[H]
    \centering
    \begin{tikzpicture}[scale=1]
        \begin{axis}[
                x label style={at={(axis description cs:0.5,-0.01)},anchor=north},
                y label style={at={(axis description cs:-0.01,0.5)},anchor=south},
                xlabel=\textbf{Nummer der Messung},
                ylabel=\textbf{Zeit in Nanosekunden},
                scaled ticks=false,
                width=1.0\textwidth,
                height=0.7\textwidth,
            ]
            \addplot table [y=$SA_F$, x=NR]{./res/data_laptop_2.dat};
            \addlegendentry{$SA_F$ series}
            \addplot table [y=$SA_H$, x=NR]{./res/data_laptop_2.dat};
            \addlegendentry{$SA_H$ series}
            \addplot table [y=$G_F$, x=NR]{./res/data_laptop_2.dat};
            \addlegendentry{$G_F$ series}
            \addplot table [y=$G_H$, x=NR]{./res/data_laptop_2.dat};
            \addlegendentry{$G_H$ series}
        \end{axis}
    \end{tikzpicture}
    \vspace*{-0.8cm}
    \caption{\textbf{2. Messreihe am Laptop (Ryzen 5 5500U)}}
\end{figure}

\clearpage

\cite{boeing_engineers}
\cite{indian_overview}
\cite{sql_ideas}
\cite{avl_tree_wikipedia}

% break urls
\emergencystretch=0.5em

\section{Quellenangaben}

\url{https://beej.us/guide/bgnet/html/}

\printbibliography[title={Literaturverzeichnis}]

\end{document}
