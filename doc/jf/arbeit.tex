% set font and paper size
\documentclass[11pt,a4paper]{article}

% change to german
\usepackage[german]{babel}

% better hyphenation
\usepackage[final]{microtype}
\usepackage{csquotes}

% packages, images, math
\usepackage{geometry, graphicx, amsmath, amsfonts, array, multicol, multirow}

% bar charts
\usepackage{bchart}

% for bibliography
\usepackage[
    backend=biber,
    style=alphabetic,
    sorting=ynt,
    minalphanames=3,
]{biblatex}
\addbibresource{./res/references.bib}

% remove dots in ToC
\usepackage{tocloft}

% import line spacing
\usepackage{setspace}

% colors
\usepackage{xcolor}

% for urls
% break urls on line end
\usepackage[breaklinks, colorlinks=true, urlcolor=blue, citecolor=blue, linkcolor=black]{hyperref}

% Remove Indentation at new line
\setlength{\parindent}{0cm}

% Set Font to Arial, needs xelatex
\usepackage{unicode-math}
\usepackage{fontspec}
\setmainfont{Arial}

% Set Layout
\geometry{
    a4paper,
    left=25mm,
    right=25mm,
    top=25mm,
    bottom=20mm
}

% plots
\usepackage{pgfplotstable}
% recommended:
\usepackage{booktabs}
\usepackage{array}
\usepackage{colortbl}

\usepackage{pgfplots}

% moving in tikz picture
\usepackage{changepage}

% position figures
\usepackage{float}

% set line spacing
\setstretch{1.5}

% remove ToC page number
\AtBeginDocument{\addtocontents{toc}{\protect\thispagestyle{empty}}} 

% document
\begin{document}

% deckblatt
% title
\title{Datenbanken und Daten - \\ Das muss passen wie Schlüssel und Schloss}
\author{Anton Rodenwald (19)}
\maketitle

% page settings
\addtocounter{page}{-4}
\thispagestyle{empty}

\large
\begin{tabular}{l p{12cm}}

    Fachgebiet:          & Mathematik/Informatik   \\

    Wettbewerbssparte:   & Jugend Forscht          \\

    Bundesland:          & Niedersachsen           \\

    Wettbewerbsjahr:     & 2024                    \\

    Thema des Projektes: & <Text>                  \\

    Projektbetreuerin:   & Birgit Ziegenmeyer      \\

    Institution:         & Schillerschule Hannover \\
\end{tabular}

\clearpage

\pagestyle{empty}

% kurzfassung

\section*{Kurzfassung}

<Text>

\clearpage

% inhaltsverzeichnis

\renewcommand*\contentsname{Inhaltsverzeichnis}

\renewcommand{\cftdot}{}

\tableofcontents

\clearpage

\pagestyle{plain}

% einleitung 
% stichpunkte?
% einfach

\section{Einleitung}
% nahbar, interesse weckend
% motivation, fragestellung

\section{Hintergrund}
% arten von daten
% arten von datenbanksystemen
% vorgehensweise

\section{Methodik}
% implementaton -> networking
% implementation -> welche algorithmen
% messung -> viele messungen, genauigkeit, unterschiedliche systeme
% messwerte


\section{Ergebnisse}
% vergleich mit graphen
% empfehlung von datenbanken -> einfach regeln
% 

% fazit
% ausblick
% interessen geweckt

% Quellen
% bjnet
% 


\clearpage

\begin{bchart}[min=0, max=30, scale=1.9]
    \bcbar[label=1, text=List Comprehension Python mit Konvertierung zu numba.typed list]{26.849}
    \smallskip
    \bcbar[label=2, text=\hspace{7cm}List Comprehension Python]{9.307}
    \smallskip
    \bcbar[label=3, text=\hspace{2cm}List Comprehension Python mit Numba]{1.148}
    \smallskip
    \bcbar[label=4, text=\hspace{2cm}NumPy np.random.randint mit Numba]{0.401}
    \smallskip
    \bcbar[label=5, text=\hspace{2cm}NumPy np.random.randint]{0.051}
    \smallskip
    \bcxlabel{\bf{Ausführungszeit in Sekunden}}
\end{bchart}

\begin{figure}[H]
    \centering
    \begin{adjustwidth*}{}{-1.2cm}
        \begin{tikzpicture}[scale=1]
            \begin{axis}[
                    x label style={at={(axis description cs:0.5,-0.01)},anchor=north},
                    y label style={at={(axis description cs:-0.01,0.5)},anchor=south},
                    xlabel=Nummer der Messung,
                    ylabel=Zeit in Nanosekunden,
                    width=1.1\textwidth,
                    height=1.6\textwidth,
                ]
                \addplot table [y=$SA_F$, x=NR]{./res/data.dat};
                \addlegendentry{$SA_F$ series}
                \addplot table [y=$SA_H$, x=NR]{./res/data.dat};
                \addlegendentry{$SA_H$ series}
                \addplot table [y=$G_F$, x=NR]{./res/data.dat};
                \addlegendentry{$G_F$ series}
                \addplot table [y=$G_H$, x=NR]{./res/data.dat};
                \addlegendentry{$G_H$ series}
            \end{axis}
        \end{tikzpicture}
    \end{adjustwidth*}
\end{figure}

\cite{finland_indepth}
\cite{indian_overview}

\section{Quellenangaben}

% break urls
\emergencystretch=0.5em

\printbibliography[title={Literaturverzeichnis}]

\end{document}
