% set font and paper size
\documentclass[11pt,a4paper]{article}

% change to german
\usepackage[german]{babel}

% better hyphenation
\usepackage[final]{microtype}
\usepackage{csquotes}

% packages, images, math
\usepackage{geometry, graphicx, amsmath, amsfonts, array, multicol, multirow}

% bar charts
\usepackage{bchart}

% check for unused references, not working
% \usepackage{refcheck}

% for bibliography
\usepackage[
    backend=biber,
    style=alphabetic,
    sorting=ynt,
    minalphanames=3,
]{biblatex}
% \addbibresource{./references.bib}

% import line spacing
\usepackage{setspace}

% colors
\usepackage{xcolor}

% for urls
% break urls on line end
\usepackage[breaklinks, colorlinks=true, urlcolor=blue, citecolor=blue, linkcolor=black]{hyperref}

% Remove Indentation at new line
\setlength{\parindent}{0cm}

% Set Font to Arial, needs xelatex
\usepackage{unicode-math}
\usepackage{fontspec}
\setmainfont{Arial}

% Set Layout
\geometry{
    a4paper,
    left=25mm,
    right=25mm,
    top=25mm,
    bottom=20mm
}

% set line spacing
\setstretch{1.5}

% document
\begin{document}

% deckblatt
% title
\title{Light DB Working Title}
\author{Anton Rodenwald (19)}
\maketitle

% page settings
\addtocounter{page}{-2}
\pagestyle{empty}

\large
\begin{tabular}{l p{12cm}}

    Fachgebiet:          & Mathematik/Informatik   \\

    Wettbewerbssparte:   & Jugend Forscht          \\

    Bundesland:          & Niedersachsen           \\

    Wettbewerbsjahr:     & 2024                    \\

    Thema des Projektes: & <Text>                  \\

    Projektbetreuerin:   & Birgit Ziegenmeyer      \\

    Institution:         & Schillerschule Hannover \\
\end{tabular}

\clearpage

% kurzfassung
\section*{Kurzfassung}

<Text>

\clearpage

% inhaltsverzeichnis
\renewcommand*\contentsname{Inhaltsverzeichnis}

\tableofcontents

% einleitung
% einleitung 
% stichpunkte?
% einfach

\section{Einleitung}
% nahbar, interesse weckend
% motivation, fragestellung

\section{Hintergrund}
% arten von daten
% arten von datenbanksystemen
% vorgehensweise

\section{Methodik}
% implementaton -> networking
% implementation -> welche algorithmen
% messung -> viele messungen, genauigkeit, unterschiedliche systeme
% messwerte


\section{Ergebnisse}
% vergleich mit graphen
% empfehlung von datenbanken -> einfach regeln
% 

% fazit
% ausblick
% interessen geweckt

% Quellen
% bjnet
% 


\section{Materialien, Vorgehen, Methode}

\subsection{Materialien}

\subsection{Vorgehen und Methode}

\subsection{Schwierigkeiten}

\section{Ergebnisse}

\subsection{Generierung von 10 Millionen zufälligen Zahlen in Python}

\clearpage

\section{Ergebnisdiskussion}

\subsection{Erklärung der Ergebnisse}

\subsection{Beantwortung der Forschungsfrage}

\section{Reflexion und Ausblick}

\section{Schlusswort}

\clearpage

\begin{bchart}[min=0, max=30, scale=1.9]
    \bcbar[label=1, text=List Comprehension Python mit Konvertierung zu numba.typed list]{26.849}
    \smallskip
    \bcbar[label=2, text=\hspace{7cm}List Comprehension Python]{9.307}
    \smallskip
    \bcbar[label=3, text=\hspace{2cm}List Comprehension Python mit Numba]{1.148}
    \smallskip
    \bcbar[label=4, text=\hspace{2cm}NumPy np.random.randint mit Numba]{0.401}
    \smallskip
    \bcbar[label=5, text=\hspace{2cm}NumPy np.random.randint]{0.051}
    \smallskip
    \bcxlabel{\bf{Ausführungszeit in Sekunden}}
\end{bchart}

\vspace{0.5cm}

% break urls
\emergencystretch=0.5em

% \printbibliography[title={Literaturverzeichnis}]

\end{document}
